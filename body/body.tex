\section{课题来源及研究的背景和意义}

\subsection{课题的来源}

项目来源:国家自然科学基金项目(编号:U1509216,U1866602)

\subsection{课题研究的背景和意义}



\section{国内外在该方向的研究现状及分析}

\subsection{国外研究现状}

\subsection{国内研究现状}


\subsection{国内外文献综述的简析}

现在相对而言,时间序列异常检测已经有了许多可行的解决方案,分为不同的方面。这些研究成果在许多领域都有了应用而且产生了不错的效益。

\section{主要研究内容}

我的毕业设计的研究内容主要从数据标签的获取难易程度入手,主要分为以下几点:

\begin{itemize}
    \item 无监督学习时间序列异常检测算法:无需数据标签就可以使用的异常检测算法,通常通过计算数据的离群程度来判断数据是否为异常点;
    \item 半监督学习时间序列异常检测算法:可以利用起有限标签的异常检测算法,因为真实的场景是我们常常可以获取到一部分先验知识,这些先验知识可以转换为部分数据的标签,利用起这部分标签能更有效地进行异常检测;
    \item 小样本学习时间序列异常检测算法:和半监督不同,这里着重研究的元学习、迁移学习等在时间序列异常检测领域的使用,在时间序列的场景下,如何利用已有的模型、知识,通过少量数据微调来获得更好的性能。
\end{itemize}

\subsection{不使用数据标签:无监督学习时间序列异常检测算法}

具有无监督学习算法是异常检测算法中使用频率最高的,现在很多研究和基准的异常检测算法都是基于无监督学习的。因为无监督学习模型在使用的时候不需要标注的数据,用起来简单,更适用于大多数场景,所以现在多数异常检测的研究方向都是在无监督学习上的。

自编码器是一种常见的神经网络模型,它可以用来做数据压缩、数据编码等,而且它天然的提取低维特征——重构原数据的模式使其能够用于异常检测领域。目前自编码器在异常检测领域已经有了一些研究,但是在时间序列的异常检测方面研究还是相对较少,而且现有的时间序列异常检测模型都相对简单,如PAA-AE,LSTM-AE和CNN-AE等。所以我考虑从这个方面入手,设计并实验验证得到一个真正适用于时间序列的异常检测自编码器模型。

考虑到现有的网络模型对时间序列的特征提取都不够:单独的RNN对时序数据的特征提取已经无法满足高维数据的要求,而非循环的神经网络又无法获得时序数据中的长期依赖,所以在未来的研究中,我想能够融合一些现在表现比较好的特征提取模型,包含时序特征的抽取和高维数据的抽取等。然后再根据这个模型构建一个自编码器用于异常检测,同时在实验验证中不断验证模型是否有更好的结果。

\subsection{使用部分标签:基于度量学习的时间序列异常检测算法}

无监督学习因为其适用性广、门槛低等优点被广泛应用,但是它的对应的归纳偏置却限制了它准确性:无监督学习异常检测算法是假定数据分布密集的部分是正常数据分布的范围,而离群点大多分布在偏离数据分布密集的位置。这种归纳偏置符合我们的直观认知,但是在一些场景下,它却也是不合理的。如风电机产生的数据中,异常包括Sparse Outliers和Stacked Outliers,其中的Stacked outliers就是聚集分布的异常点,使用传统的无监督异常检测方法会将这些数据误判为正常点。但是其实此时我们是知道这部分点的异常情况的,无监督学习方法无法利用这些知识帮助检测异常。

所以,如果能够利用起少量的数据标签用于辅助异常点的决策,如在前面的例子中,能否利用起已知的Stacked Outliers先验,告诉模型这部分堆积的数据也是异常数据呢?我们知道,监督学习是能够利用起数据中的标签的,但是,监督学习面向的是标签的数据,我们现在仅有部分标签和先验知识,如果只在这些数据上使用监督学习的话,又会使模型过拟合于这部分数据。所以,我们需要既利用起有限的标签和知识,也要利用起大量的无标签数据来辅助对异常点的判断,这其实就是半监督学习的思路。

想到半监督学习的异常检测,我们很容易就能想到使用半监督的分类模型来解决,将正常和异常的样本视为两类,转换为一个二分类问题。但是实际上,异常检测问题并不能简单地被归纳为二分类问题,因为通常情况下异常数据的特征是不可控的,即我们无法通过一种统一的模式来识别所有的异常。通常在异常检测领域的做法是,转换为单分类问题(one-class classification)。

基于这一点入手,我未来会从度量学习入手。原来的度量学习是针对分类问题,将数据映射到一个理想的嵌入空间上。嵌入的规则就是使相同类比的数据进行成簇分布。在异常检测方面,和分类问题会有很大的不同。暂定的想法是,使所有的正常样本在嵌入空间中尽量地聚簇分布,而使异常样本距离正常样本的距离尽量地远,通过神经网络将数据映射到这样的一个嵌入空间中,从而根据嵌入的数据到正常簇的距离,判断数据是否异常。这个过程需要适合的网络模型设计和损失函数的设计。同时,为了利用起无标签的数据,可以利用度量学习模型对现有的数据集进行扩容,为无标签的数据打上不同置信度的标签,不断地迭代优化模型。

而且,基于度量学习的时间序列异常检测算法还可以引申到在线学习的检测。因为某些适合会需要实时的时间序列异常检测,所以在线算法也是有很大的应用空间的。这些都会在未来的研究中体现。

\subsection{小样本学习:基于元学习的时间序列异常检测算法}

仅有少量样本的场景下,小样本学习(few-shot learning)是一个很好的选择。其实从某种程度来说,前面的半监督学习方法也可以归为小样本学习的一种。但这部分和半监督学习不同,这里使用少量样本的方式是元学习、迁移学习这些。这些方法在图像、音视频领域都有了很多研究,但是在时间序列领域还是鲜有研究。而现在工业数据越来越多,小样本的场景也越来越多,如何将这些方法应用到时间序列的异常检测领也是十分重要的。

元学习的目标是在多个任务的训练中,在新的任务执行前,能够根据任务的数据,找到一个最适合该任务的参数初始化方式,然后再根据数据的少量标签对模型进行微调,使模型能够迅速地收敛到新的模型和数据上。因为基于元学习的时间序列异常检测模型很少,所以未来首先要进行调研,找一些合适的任务集合和网络模型(或者使用前面的半监督学习模型),从而能够使模型在只有少量数据的新任务上,也能产生好的结果。

\section{已完成的研究工作}

\begin{itemize}
    \item[(1)] 无监督学习异常检测方面,现在已经有了成型了思路,也做了一些实验验证,模型是有效果的;
    \item[(2)] 度量学习方面,目前已经完成了调研,也有了基本的思路。基础特征提取模型也已经形成,架构上还需要根据实验结果进一步做调整;
    \item[(3)] 元学习方面,现在还处于调研和思考的阶段,方向已经确定,但是如何实现、实现到什么程度还是要在未来进行调整;
    \item[(4)] 实验框架设计也已经有了成型的思路,现在的研究内容是有递进关系的,实验过程也会对不同的方法进行对比。同时也找了一些合适的时间序列异常检测数据集,包含低维、高维、公开、工业界各个方面。
\end{itemize}

\section{研究方案及进度安排,预期达到的目标和取得的研究成果}

\subsection{研究方案}

\begin{itemize}
    \item[(1)] 完成无监督学习已有的想法,进一步进行实验,优化融合模型的自编码器上。整理总结相关的文档,形成完整的模型;
    \item[(2)] 进一步调研学习度量学习,基于不同的网络模型,测试提出的度量学习异常检测损失函数,调整权重完善模型。同时在不同的数据集上进行实验并可视化,验证算法的有效性。
    \item[(3)] 扩展半监督学习异常检测模型,提出有效的数据标签扩充方案,并设置伪标签的置信度计算策略,挑选置信度高的伪标签数据进一步优化模型;
    \item[(4)] 将基于度量学习的半监督异常检测算法扩展到在线学习,提出增量学习模型的学习策略,包含数据扩充策略以及参数增量更新策略等;
    \item[(5)] 调研元学习在时间序列的应用,设计网络和损失函数,使用合适的模型在异常检测数据集上进行实验测试,在小样本的情况下向新场景扩展,确保模型的稳定性和性能;
    \item[(6)] 在业界公开和工业界电网数据集上进行测试实验,计算模型进行异常检测的准确率、召回率、AUC等值,和已有的baseline和最新的算法进行比较,同时验证模型在不同的场景下是否有预期的收敛行为,观察自编码器的重构效果、度量学习的嵌入空间分布等; 
    \item[(7)] 形成一个统一的架构,根据知识和标签的拥有比例,为使用者提供合适的异常检测算法。
\end{itemize}

\subsection{进度安排、预期达到的目标}

进度安排如表\ref{table}

\begin{table}[]
    \caption{进度安排}
    \label{table}
    \resizebox*{\textwidth}{44mm}{
    \begin{tabular}{@{}cc@{}}
        \toprule
    起止时间          & 进度安排                            \\ \midrule
    现在——11月中旬     & 阅读论文,进一步完善想法,完成无监督学习异常检测模型,测试指标 \\
    11月中旬——秋季学期结束 & 确定度量学习的思路,代码实现度量学习异常检测模块        \\
    秋季学习结束——中期答辩  & 实验验证并调整度量学习异常检测模块,形成最终版本,测试指标   \\
    中期答辩——四月中旬    & 补充元学习相关的知识,整理出可行的异常检测思路         \\
    四月中旬——五月中旬    & 代码实现元学习相关的内容,测试指标,总结文档          \\
    五月中旬——结题答辩    & 统一整理三方面研究的内容,形成一个整体的异常检测系统        \\ \bottomrule
    \end{tabular}}
\end{table}

\section{为完成课题已具备和所需的条件和经费}

\begin{itemize}
    \item[(1)] 知识储备方面,现在已经阅读了一系列的文献,对无监督学习、半监督学习算法有了一定的了解,而且无监督学习已经实现了一些算法,后期只需要调试和验证即可;
    \item[(2)] 硬件方面,能够使用海量数据研究中心的集群;
    \item[(3)] 数据方面,目前有和国家电网合作的工业数据集,同时也准备了yahoo、UCR等时间序列公开数据集供未来实验使用;
    \item[(4)] 软件方面,torch框架现在已经熟悉,而且实现了一些自编码器模型;
    \item[(5)] 经费方面,研究过程中能使用到经费的地方不多,大多是一些文献的下载、书籍的购买等,以及一些计算资源的使用。
\end{itemize}

\section{预计研究过程中可能遇到的困难和问题,以及解决的措施}

\begin{itemize}
    \item[(1)] 网络模型的挑选和参数的选择,这部分经验较为欠缺,如何最好的发挥模型的作用需要针对不同的场景进行参数调整,这一点需要不断试验;
    \item[(2)] 标签数据较难获得,这部分不是要标签数据来在检测适合使用,而是验证的过程中使用。因为公开的标签异常检测数据集较少,所以可能会需要找一些二分类数据集;
    \item[(3)] 元学习现在的了解程度还不够,还需要进一步的阅读相关的论文和文档,如何划分数据集,如何训练、迁移,都是需要学习的地方;
    \item[(4)] 半监督学习过程中因为数据扩充是根据已有的标签数据,所以很可能会过拟合这部分数据,如何根据无标签数据的分布解决过拟合问题也是比较困难的研究点;
\end{itemize}



aaa\cite{hithesis2017}

\bibliographystyle{hithesis}